\documentclass[a4paper,12pt]{article}
\usepackage[left=2.5cm, right=2.5cm, top=1.5cm, bottom=1cm]{geometry} %layout
\usepackage[utf8]{inputenc}
\pagenumbering{gobble}

\title{Capstone Project Exposé - Draft 2}
\author{Martin Zaubitzer }
\date{\today}

\begin{document}

\maketitle

\section*{Meta-Info}
\begin{itemize}
    \item Supervisor: Ulrich von Zadow
    \item Preliminary Capstone Project hand-in deadline: end November 2022
\end{itemize}

\section*{Introduction}
The climate crisis is a topic often discussed in todays politics. Within these discussions the CO\textsubscript{2} emissions from cars and cargo transportation are sometimes the focus. A very efficient and enviromental friendly way to transport cargo over long distances, is transportation via rail, which is now getting more attention as a possible replacement for long-haul trucks. To insure safety and financial feasibility for the companies in this industry, a fleet of reliable, well maintained locomotives and freight waggons is required. While most locomotives are fairly new and technologically advanced, most freight carts used in Germany are old and without any build-in electrical sensors. Therefore maintanace is complicated and requires specially trained professionals with very specific equipment. While specialized equipement is being develop regularly, finding these professionals or other workers to train gets harder ervery day.

\section*{Goals}
To combat this, we have decided to rebuild this maintenance process from the ground up and create a device (and toolchain) that can automatically test and diagnose freight waggons. The goal is that this process can be used by every worker in the field, without needing special training.

\noindent
Multiple prototypes have already been created for internal testing. Now we are ready to develop a final prototype of the automated freight waggon testing device, that can be presented to the audience of Innotans 2022.

\noindent
My goal at the end of this project is, to have a presentable prototype of the device, that can automatically and manually test the freight waggon and generate a digital report. This includes a demo Web Interface, simulating the office enviroment where work requests are submitted, an Android Application to remotely control the device and genereate the report and the device with all the sensors and manual controls.

\newpage

\section*{Project Outcome}
The project outcome will be split into three different artifacts:
\begin{description}
    \item[1. Web (Admin) Interface:] \hfill \\
        This will be a front-end Web Interface where work requests can be submitted and workers dispatched. In the back-end it will hold an example database of known freight waggons, users and work results.
        It will be built using C\# (Blazor/.NET) or golang.
    \item[2. Android Application:] \hfill \\
        This will be the main part of the project, a full Android application that can fetch incomming work requests for a specific worker. It will include Ble to communicate with the testing device and controll it. In the end, the App will send the test result back to the back-end and create a PDF work result that can be directly send to the customer.
        It will be built using Kotlin and Java.
    \item[3. Testing Device:] \hfill \\
        This will be a microcontroller/Arduino/Controllino based device with Ble capabilities with sensors and valves to measure and control the freight waggon. These waggons work with air pressure, so the device will meassure existing pressure and fill/empty the waggon using an external compressor. It will also have a display and physical buttons to manually control all actions directly on the device without the need of the Android Application.
        It will be built using Arduino via platformIO.
\end{description}
\noindent
I would consider this project successfull when all three artifacts can be demonstrated during a live demo.

\section*{Project Plan}
To arrive at the final outcome of this Capstone project, it will implement the following steps:
\begin{description}
    \item[1. Evaluation the current progress:] \hfill \\
        The progress made so far will be deeply analyzed to find where to improve the next device/app generation. In this step it will also be determined which of the multiple existing guidelines to follow and how.
    \item[2. Clean-up and Rework of the Device Code:] \hfill \\
        The existing code for the device will be reworked and tested.
    \item[3. Clean-up and Refactor of the Android Application:] \hfill \\
        The existing application will undergo massive restructering and testing.
    \item[4. New feature implementation:] \hfill \\
        After all parts are cleaned and tested the new features (determined by Step 1) will be implemented.
    \item[5. Develop the Web Interface for demonstration:]
        The back-end databases and Admin/Web front-end will be built so simulate the full system.
\end{description}

\section*{Organizational Setup}
This project will be implemented as a solo CODE semester project. I will develop and design all three artifacts alone. To get done until the demonstration some example data will be fictional and a co-worker will design and assemble the device from a pneumatic/electronics perspective.

\section*{Roadmap}
\begin{description}
    \item[Now - End of June:] \hfill \\
        Submission of this Project.
        \item[June - September:]\hfill \\
        Current State evaluation and development of new features.
    \item[20.09.2022:] \hfill \\
        Customer demonstration at Innotrans 2022.
        \item[September-November:]\hfill \\
        Finish the project development and start development of the web/back-end elemnts. Final testing and clean-up.
    \item[End of November]
        Ready for complete project demonstration.
\end{description}

\section*{Proposed Assessment Modalities}
The final examination will be based on a live demonstration of the project and/or an oral examination.

\noindent
The live demo will show the project from a users perspective and therefore will focus on:
\begin{itemize}
    \item ease of use for an untrained user
    \item the full process from job generation in the Web Interface to the result generation in the Android App (maybe as pre-recorded video)
    \item automated/remote control via the App and manual control directly on the device
\end{itemize}

\end{document}
