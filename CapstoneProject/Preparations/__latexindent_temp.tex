\documentclass[a4paper,12pt]{article}
\usepackage[left=2.5cm, right=2.5cm, top=1.5cm, bottom=1cm]{geometry} %layout
\usepackage[utf8]{inputenc}
\pagenumbering{gobble}

\title{Capstone Project - Short Project Description}
\author{Martin Zaubitzer }
\date{March 2022}

\begin{document}

\maketitle

\section*{Introduction}
Many things small and big things are currently developed, trying to combat the climate crisis. One of these things, in regards to long distance cargo transportation, could be the shift from trucks to rail transportation. To insure safety and financial feasibility for the companies in this industry, a fleet of reliable, well maintained and cheap locomotives and freight carts is required. While most locomotives are fairly new and technologically advanced, most freight carts are very old and without any electrical sensors. This makes maintenance complicated and requires a lot of time, specially trained personnel and equipment. \\

\noindent To combat this, the company I am currently with has decided to rebuild this maintenance process from the ground up and create a device that can automatically test and diagnose freight carts. Multiple prototypes have been created and now we think the device is ready for a final iteration. I will be developing this final iteration of freight cart maintenance device, in a solo project, from a software perspective. It will also include the development of an companion application allowing for remote control, data aggregation and report creation. 

\section*{Project Outcome}
The outcome of this project will be a pneumatic device that can measure and adjust the pressure from a specific freight cart. It can also display the results directly to a user. Besides that, it will also be able to run a number of predefined test. Along-side it will have an Android Companion App, able to remote control the device, run more sophisticated automated and manual tests. The App will also be able to take result data from the device, generate a PDF report and make the data available to an company internal API.
\\
\noindent The device and its App should able to run a predefined list of automated and manual tests at the end of this project, for it to be successful. Should it turn out that data sharing to internal or external APIs is not wanted or needed this project would still be a success.

\newpage

\section*{Project Plan}
To arrive at the desired outcome of this Capstone project, it will implement the following steps:
\begin{description}
    \item[1.Evaluation of current progress:] \hfill \\ 
        The progress made so far will be deeply analyzed to find where to improve the next device/app generation. In this step it will also be determined which of the multiple existing guidelines to follow and how.
    \item[2.Field Tests and comparisons:] \hfill \\ 
        With the current prototype evaluated it is time to test efficiency and reliability. In this step it is also time to compare the automated test in their efficiency compared to traditional testing. 
    \item[3. Improvement analysis and Device and App Creation:] \hfill \\ 
        When all this data is collected it is time to start analyzing where a new iteration of device/app could improve upon the existing one. A new device will than be build from the ground up to implement the changes and new features and dial in the latest changes. The app will also be reworked to accommodate the device changes and bring more value to the overall concept.  
    \item[4. Field Tests and analysis:] \hfill \\ 
        After all changes are implemented, the new device will be tested against the previous one and the traditional way of testing. This will also include user tests and more feedback that will greatly help the project for the future.
\end{description}

\section*{Organizational Setup}
As stated this project is a work project, where I am leading the software development of the device and the Android Application. The team working at this in the company consists of two other people, responsible for device design, pneumatic and electronics. I will be working alone on the software parts, but try to peek into other parts if necessary. If possible, I will not do this project in the CODE semester framework.

\section*{Roadmap}
\begin{description}
    \item[Now - End of April:] \hfill \\
    Gather all requirements and make the project official.  
    \item[April - June:]\hfill \\
    Finish all testing and evaluation on the old device and start development of the new under this project.
    \item[June-August:]\hfill \\
    Finish the project development and start testing the new device, including user tests.
\end{description}

\end{document}
